\documentclass{beamer}

% Tema sencillo
\usetheme{Warsaw}

\title{Leyes de Newton}
\author{Ph.D(c) Informática \,|\, MSc. Física \,|\, Lic. Física \,|\, Ing. Eléctrico}
\date{\today}

\begin{document}

\begin{frame}{Leyes de Newton}
\textbf{1. Ley de la inercia} \\
Un cuerpo permanece en reposo o en movimiento rectilíneo uniforme si la fuerza neta sobre él es cero.

\vspace{0.4cm}
\textbf{2. Ley fundamental de la dinámica} \\
\[
\vec{F}_{\text{net}} = m \, \vec{a}
\]

\vspace{0.4cm}
\textbf{3. Ley de acción y reacción} \\
Si un cuerpo A ejerce una fuerza sobre un cuerpo B, entonces B ejerce una fuerza de igual magnitud y dirección opuesta sobre A.
\end{frame}

\begin{frame}{Aplicaciones de las Leyes de Newton}
	\textbf{Ejemplo 1: Cuerpo sobre una superficie horizontal} \\[4pt]
	Un bloque de masa \( m \) es empujado con una fuerza \( F \) sobre una superficie con fricción \( f \).  
	La segunda ley de Newton establece:
	\[
	\Sigma F_x = F - f = m a
	\]
	
	\vspace{0.4cm}
	\textbf{Ejemplo 2: Caída libre} \\[4pt]
	Si se deja caer un objeto desde cierta altura, la única fuerza que actúa sobre él es el peso:
	\[
	F = m g
	\]
	por lo tanto, la aceleración es constante y igual a \( g \approx 9.8\,\text{m/s}^2 \).
	
	\vspace{0.4cm}
	\textbf{Conclusión:} Las leyes de Newton permiten predecir y explicar el movimiento de los cuerpos bajo distintas fuerzas externas.
\end{frame}

\end{document}
