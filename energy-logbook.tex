\documentclass{beamer}

% Tema sencillo
\usetheme{Madrid}

\title{Leyes de Newton}
\author{Ph.D(c) Informática \,|\, MSc. Física \,|\, Lic. Física \,|\, Ing. Eléctrico}
\date{}

\begin{document}

\begin{frame}{Leyes de Newton}
\textbf{1. Ley de la inercia} \\
Un cuerpo permanece en reposo o en movimiento rectilíneo uniforme si la fuerza neta sobre él es cero.

\vspace{0.4cm}
\textbf{2. Ley fundamental de la dinámica} \\
\[
\vec{F}_{\text{net}} = m \, \vec{a}
\]

\vspace{0.4cm}
\textbf{3. Ley de acción y reacción} \\
Si un cuerpo A ejerce una fuerza sobre un cuerpo B, entonces B ejerce una fuerza de igual magnitud y dirección opuesta sobre A.
\end{frame}

\end{document}
